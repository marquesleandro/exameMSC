\usepackage[T1]{fontenc} %  oriented to output, encoding of fonts to use for printing characters
\usepackage[utf8]{inputenc} % oriented to input accented characters directly from the keyboard
\usepackage[main=portuguese]{babel} % to use English as main language;
\usepackage{enumerate} % to use enumerate environmente, when producing lists
\usepackage{cite} % for improved handling of numeric citations. 
\usepackage{epsf,epsfig,psfig} % to insert '.eps', ' .ps' postscript figures
\usepackage{indentfirst} % to indent paragraphs
\usepackage{pagina} % UERJ formatting
\usepackage{theorem} % to use theorem environment
\usepackage{fancyhdr} % to get better control over headers
\usepackage{setspace} % to specify line spacing for sections and paragraphs through built-in commands 
\usepackage{boxedminipage} % to produce framed minipages
\usepackage{float} % to float figures and tables
\usepackage{makeidx} % to organize indexing
\usepackage{amsmath,amssymb,bm,amsbsy} % to use a plethora of mathematical AMS fonts
\usepackage{mathtools} % extension to package 'amsfonts' for bug correction and mathematical typesetting
\usepackage[hidelinks]{hyperref} % to enable typesetting of hyperlinks; 'hidelinks': remove color and border
\usepackage{subfig} % to use subfigure environment
\usepackage{algorithm2e} % to use algorithm environment
\usepackage{textcomp} % to use the Text Companion fonts
\usepackage[left]{lineno} % to have line numbering for referral and review purposes. Use option 'switch' to have odd/even numbering; 'pagewise' to have numbering per page.
\usepackage[compatible, english]{nomencl} % to produce nomenclature. Use 'portuguese' for Portuguese title.
\usepackage{booktabs, longtable} % to have nice tables 
\usepackage{blkarray} % to have labeled rows/columns in matrices
\usepackage{cleveref} % to configure 'clever' references
\usepackage{upgreek} % to have upright style for \greek fonts 
\usepackage{cjhebrew} % to insert hebrew characters
\usepackage{morefloats} % to force Latex to cope with larger number of floats
\usepackage{multirow} % to allow multirows in tables
\usepackage{footnote} % to add general footnotes
\usepackage[referable]{threeparttablex} % to create partite tables
\usepackage[bottom]{footmisc} % to force footnotes at the bottom of the page
\usepackage{lipsum} % dummy text
\usepackage{microtype} % text kerning (ajuste de espaço de caracter) for better visual. LOAD THIS PACKAGE LAST!
